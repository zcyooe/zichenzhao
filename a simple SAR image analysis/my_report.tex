% !TEX TS-program = pdflatex
% !TEX encoding = UTF-8 Unicode

% This is a simple template for a LaTeX document using the "article" class.
% See "book", "report", "letter" for other types of document.

\documentclass[11pt]{article} % use larger type; default would be 10pt

\usepackage[utf8]{inputenc} % set input encoding (not needed with XeLaTeX)

%%% Examples of Article customizations
% These packages are optional, depending whether you want the features they provide.
% See the LaTeX Companion or other references for full information.

%%% PAGE DIMENSIONS
\usepackage{geometry} % to change the page dimensions
\geometry{a4paper} % or letterpaper (US) or a5paper or....
% \geometry{margin=2in} % for example, change the margins to 2 inches all round
% \geometry{landscape} % set up the page for landscape
%   read geometry.pdf for detailed page layout information

\usepackage{graphicx} % support the \includegraphics command and options

% \usepackage[parfill]{parskip} % Activate to begin paragraphs with an empty line rather than an indent

%%% PACKAGES
\usepackage{booktabs} % for much better looking tables
\usepackage{array} % for better arrays (eg matrices) in maths
\usepackage{paralist} % very flexible & customisable lists (eg. enumerate/itemize, etc.)
\usepackage{verbatim} % adds environment for commenting out blocks of text & for better verbatim
\usepackage{subfig} % make it possible to include more than one captioned figure/table in a single float
% These packages are all incorporated in the memoir class to one degree or another...

%%% HEADERS & FOOTERS
\usepackage{fancyhdr} % This should be set AFTER setting up the page geometry
\pagestyle{fancy} % options: empty , plain , fancy
\renewcommand{\headrulewidth}{0pt} % customise the layout...
\lhead{}\chead{}\rhead{}
\lfoot{}\cfoot{\thepage}\rfoot{}

%%% SECTION TITLE APPEARANCE
\usepackage{sectsty}
\allsectionsfont{\sffamily\mdseries\upshape} % (See the fntguide.pdf for font help)
% (This matches ConTeXt defaults)

%%% ToC (table of contents) APPEARANCE
\usepackage[nottoc,notlof,notlot]{tocbibind} % Put the bibliography in the ToC
\usepackage{amsmath}
\usepackage{graphicx}
\usepackage{CJK}
\usepackage[titles,subfigure]{tocloft} % Alter the style of the Table of Contents
\renewcommand{\cftsecfont}{\rmfamily\mdseries\upshape}
\renewcommand{\cftsecpagefont}{\rmfamily\mdseries\upshape} % No bold!

%%% END Article customizations

%%% The "real" document content comes below...

\title{Operational Statistics for SAR Imagery Course Assignment}
\author{Zi Chen Zhao19171213824}
%\date{} % Activate to display a given date or no date (if empty),
         % otherwise the current date is printed 
\begin{document}
\maketitle

\section{Different Statistical Distribution}

In this course, we mainly learned about the distribution of statistical models of Synthetic Aperture Radar (SAR) imagery, such as Gamma distribution, K distribution, etc. At the same time, we also study to use different tools to plot these distribution.

\subsection{Exponential Distribution}

Firstly,we learned the exponential distribution.The distribution function is:
\begin{equation}
f(x) = \frac{1}{\sigma^2}e^\frac{-x}{\sigma^2}
\end{equation}
\\I plot the distribution with means 1/2,1 and 2(red,black,blue,resp.),then convert it to logarithmic coordinates.The plot is shown as Figure1:
\begin{figure}[htb] \center
\includegraphics[width=5cm]  {C:/Users/asus/Desktop/fa.png}
\includegraphics[width=5cm]  {C:/Users/asus/Desktop/fa1.png}
\caption{\label{1} Exponential distribution of Cartesian coordinates(left) and logarithmic coordinates(right)} 
\end{figure}
\subsection{Gamma Distribution}
Secondly,we learned the  Gamma distribution.The distribution function is:
\begin{equation}
f_Z(z,L,\sigma^2) = \frac{L^L}{\sigma^{2L}\Gamma(L)}z^{L-1}exp\{-Lz/\sigma^2\}
\end{equation}
where $\Gamma(v)$ is the Gamma function given by $\Gamma(v) = \int_{R_+} {t^{v-1}e^{-t}dt}$
I plot three  cases of the Gamma distribution with uni-tary mean and shape parameters (Looks) equal to 1 (the Exponen-tial distribution), 3 and 8,then convert it to logarithmic coordinates.The plot is shown as Figure 2:
\begin{figure}[htb] \center
\includegraphics[width=5cm]  {C:/Users/asus/Desktop/f2.png}
\includegraphics[width=5cm]  {C:/Users/asus/Desktop/f21.png}
\caption{\label{1} Gamma distribution of Cartesian coordinates(left) and logarithmic coordinates(right)} 
\end{figure}
\subsection{K Distribution}
The K distribution function is:
\begin{equation}
f_Z(z,\alpha,\lambda,L) = \frac{2{\lambda}L}{\Gamma(\alpha)\Gamma(L)}{\lambda}Lz^{\frac{\alpha+L}{2}-1}K_{\alpha-L}(2\sqrt{\lambda{Lz}})
\end{equation}
where $\alpha>0$ measures the roughness,$\lambda>0$ is a scale parame-ter, and $K_v$ is the modified Bessel function of order $v$ . This special function is given by $K_v(z) = \int_0^\infty {e^{-z}cosh(vt)dt}$.
I try to plot the K distributions with unitary mean ( $\alpha \in { 1,3,8 }$ in red, blue, black, resp.),then convert it to logarithmic coordinates.The plot is shown as Figure3:
\begin{figure}[htb] \center
\includegraphics[width=5cm]  {C:/Users/asus/Desktop/dki.png}
\includegraphics[width=5cm]  {C:/Users/asus/Desktop/dki1.png}
\caption{\label{1} K distribution of Cartesian coordinates(left) and logarithmic coordinates(right)} 
\end{figure}
\subsection{$G_0$ Distribution}
The $G_0$ distribution function is:
\begin{equation}
f_Z(z,\alpha,\gamma,L) = \frac{L^L\Gamma(L-\alpha)}{\gamma^\alpha\Gamma(L)\Gamma(-\alpha)}\frac{Z^{l-1}}{(\gamma+Lz)^{{L-\alpha}^\prime}}
\end{equation}
I try to plot the $G_0$ distributions with unitary mean ( $\alpha \in { -1.5,-3,-8 }$ in red, blue, black, resp.),then convert it to logarithmic coordinates.The plot is shown as Figure4:
\begin{figure}[htb] \center
\includegraphics[width=5cm]  {C:/Users/asus/Desktop/fz.png}
\includegraphics[width=5cm]  {C:/Users/asus/Desktop/fz1.png}
\caption{\label{1} $G_0$ distribution of Cartesian coordinates(left) and logarithmic coordinates(right)} 
\end{figure}
\section{SAR Image Analysis}
In this part, I selected a 70*154 size image from the given SAR images.The selected image is shown in figure5 :
\begin{figure}[htb] \center
\includegraphics[width=5cm]  {C:/Users/asus/Desktop/test_image.png}
\caption{\label{1} The selected SAR image} 
\end{figure}
For the selected image, I first converted it to a grayscale image and plotted the histogram.The histogram of this image is shown as figure6:
\begin{figure}[htb] \center
\includegraphics[width=10cm]  {C:/Users/asus/Desktop/gray.png}
\caption{\label{1} The histogram of selected SAR image} 
\end{figure}
According to the histogram I can observe that it Approximately obey the Gamma distribution.To confirm this idea, I tried to plot the Gamma distribution and adjusted the parameters to make the distribution line'shape better fit.After many different experiments, I finally chose the shape parameter being 4 and the scale parameter being 1/12, 1/11 and 1/10, respective.The final result is shown as figure7:
\begin{figure}[htb] \center
\includegraphics[width=10cm]  {C:/Users/asus/Desktop/results.png}
\caption{\label{1} The final result} 
\end{figure}
\end{document} 
pdflatex example1.tex

